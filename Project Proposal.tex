%%%%%%%%%%%%%%%%%%%%%%%%%%%%%%%%%%%%%%%%%
% Short Sectioned Assignment
% LaTeX Template
% Version 1.0 (5/5/12)
%
% This template has been downloaded from:
% http://www.LaTeXTemplates.com
%
% Original author:
% Frits Wenneker (http://www.howtotex.com)
%
% License:
% CC BY-NC-SA 3.0 (http://creativecommons.org/licenses/by-nc-sa/3.0/)
%
%%%%%%%%%%%%%%%%%%%%%%%%%%%%%%%%%%%%%%%%%

%----------------------------------------------------------------------------------------
%	PACKAGES AND OTHER DOCUMENT CONFIGURATIONS
%----------------------------------------------------------------------------------------

\documentclass[paper=a4, fontsize=11pt]{scrartcl} % A4 paper and 11pt font size

\usepackage[T1]{fontenc} % Use 8-bit encoding that has 256 glyphs
\usepackage{fourier} % Use the Adobe Utopia font for the document - comment this line to return to the LaTeX default
\usepackage[english]{babel} % English language/hyphenation
\usepackage{amsmath,amsfonts,amsthm, amssymb} % Math packages
\usepackage{lastpage}
\usepackage{lipsum} % Used for inserting dummy 'Lorem ipsum' text into the template

\usepackage{sectsty} % Allows customizing section commands
\allsectionsfont{\centering \normalfont\scshape} % Make all sections centered, the default font and small caps

\usepackage{fancyhdr} % Custom headers and footers
\pagestyle{fancyplain} % Makes all pages in the document conform to the custom headers and footers
\fancyhead{} % No page header - if you want one, create it in the same way as the footers below
\fancyfoot[L]{CS364} % Empty left footer
\fancyfoot[C]{Final Project} % Empty center footer
\fancyfoot[R]{\thepage\ of \pageref{LastPage}} % Page numbering for right footer
\renewcommand{\headrulewidth}{0pt} % Remove header underlines
\renewcommand{\footrulewidth}{0pt} % Remove footer underlines
\setlength{\headheight}{13.6pt} % Customize the height of the header

\numberwithin{equation}{section} % Number equations within sections (i.e. 1.1, 1.2, 2.1, 2.2 instead of 1, 2, 3, 4)
\numberwithin{figure}{section} % Number figures within sections (i.e. 1.1, 1.2, 2.1, 2.2 instead of 1, 2, 3, 4)
\numberwithin{table}{section} % Number tables within sections (i.e. 1.1, 1.2, 2.1, 2.2 instead of 1, 2, 3, 4)

\setlength\parindent{0pt} % Removes all indentation from paragraphs - comment this line for an assignment with lots of text

%----------------------------------------------------------------------------------------
%	TITLE SECTION
%----------------------------------------------------------------------------------------

\newcommand{\horrule}[1]{\rule{\linewidth}{#1}} % Create horizontal rule command with 1 argument of height

\title{	
\normalfont \normalsize 
\textsc{CS364: Artificial Intelligence - Project Proposal} \\ [25pt] % Your university, school and/or department name(s)
\horrule{0.5pt} \\[0.4cm] % Thin top horizontal rule
\huge Baseball Pitchers Decision Tree\\ % The assignment title
\horrule{2pt} \\[0.5cm] % Thick bottom horizontal rule
}

\author{Drew Gotbaum, Eli Rosenberg, Mitch Novak, and Dennis Kachintsev} % Your name

\date{\normalsize\today} % Today's date or a custom date

\begin{document}

\maketitle % Print the title

%----------------------------------------------------------------------------------------
%	PROBLEM 1
%----------------------------------------------------------------------------------------
\section*{Abstract}
\begin{description}
\item[Project Type:] Programming
\item[Statement:] Given a dataset of training data, we would like to predict the success of any given pitcher over the course of their career to come.
\item[General Approach:] Having collected and manipulated a dataset of training data, we will construct a decision tree  to determine the pitcher's success on a year-by-year basis. We will be able to estimate future performance.
\item[Data:] There is an abundance of baseball databases available for our program, but we have not determined which is the optimal data set given our project's demands.
\item[Evaluation:] We will be using past player performance to determine their furure performances that are already known to produce a margin of error.
\item[Background Reading:] \hfill \begin{itemize}
 \item[$\bigstar$] FiveThirtyEight is statistician Nate Silver's blog consisting of a very resourceful sports section with different approaches to predicting different data points of the game.
 \item[$\bigstar$] The Bill James Historical Baseball Abstract offers different revolutionary formulas for predict the prduction of a player and the success of a team.
 \end{itemize}
\end{description}
\section*{The Team}
\begin{description}
\item[Research (November 21)] Look into the history of baseball prediction, similar artificial intelligence projects within the field, and the many sports blogs for potential formulaic ideas.
\item[Architect (November 25)] Given the datasets at our disposal, the architect must construct object oriented data structures to easily hold, manipulate and interact with the data.
\item[Statistician (November 27)] Analysis of the research, which will be used to help structure the formulas used within our algorithm.
\item[Algorithm Design (December 12)] Construct the optimal decision tree to work alongside the 'ideal' algorithm that we will choose. From the decision tree, expand the algorithm to take our tree as input.
\item[Testing Engineer (December 15)] Provide multiple test cases, consisting of edge cases, recent players, and 'outlying' players.
\end{description}






\framebox[\textwidth]{We have adhered to the honor code on this assignment.} 






\end{document}